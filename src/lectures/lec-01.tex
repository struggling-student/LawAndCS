\lesson{}{Determinism - Free Will - Combinationism}

\subsection{Power and Data Lordship}

\subsubsection{Data power}

This kind of power is neither deterministically configured nor qualified by free will,
but occurs in the impersonal concretization of the Power of automation. 

When the data related to human networkedd movements have been processed, only their 
automated realization can be given, which neither violates nor respects \textbf{determinism}
or \textbf{free will}. 

Concernig law, the compounding and traversing, in an inexhaustible circularity, of these two
powers show that what had hitherto\unsure{Hitherto, errore di battitura?} been named as a juridical
judgment is manifested as a combinatorics of data issued by the robocourt, structured as the 
robolegislator and the robo-philosopher (so as a legal/formal judgement).
\begin{note}
    These three figures are configured neither by determinism nor by free will, but currently belong
    to what is being delineated as \textbf{combinationism}, elucidated \textbf{Determinism-Free 
    Will-Combinationism} is the structural affinity with the \textbf{clinamen}, which Lucretius - \textit{De rerum natura} -
    presents as the movement of deviation of atoms in their process of falling into a vertical line.
\end{note}

\begin{definition}[What does a Clinamen mean?]
    The \textbf{clinamen} is \textbf{is a movement governed neither by the laws of determinism nor by any mode of free will}.

    It shrinks from an already predetermined whole and yet does not belong to a free will. Illuminated
    even in its poetic exposition, the clinamen can enable us to grasp that condition of human beings 
    which is not consigned to the executive passivity of already determined operations, but neither is
    is called to the creative activity of a free production of the laws of physics, chemistry, neurobiology,
    mechanics, etc.
\end{definition}

in today's data civilization, which now sets the existence of human beings, the reference to \textbf{clinamen}
can shed light on combinationalism, which names the composing of data with the production of other data,
in an itinerary where the algorithmic knowledge of the networked movements of human beings, navigator-users,
leads to a new condition, governed neither by determinism nor by free will, but configured by the \textbf{Power of data}.

It is the condition resulting from an integral combinatorics of the information elements, which does not await
the free choices and decisions of human beings, nor does it executre deterministic laws, but replaces
them with the succession of computational operations, concretized by the Power of automation, which imposes
depersonalized movement on humang beings, who have become the bystanders, the spectators of combinatorism.

As such, in law, women and men are the recipients of the enunciations, of the judgments, issued by the robocourt,
structured as the immune system of the functional success of the operations generated by the informational unity
of the Power of data and the Power of automation, operational according to the Technologies of Information and Connections(TIC).

They are the human elements not reducible to a calculation of data, formative of a \textbf{predictive computationality}, 
destined, without sufficient legal reason, either to an algorithmic elaboration of legal rules (legislative-robust activity), or
to their automated concretization (jurisdictional-robust activity).

\begin{note}
    Progressively pursued is the more efficient, robotic-informational automation of production operations, exemplified
    by the assembly line, effectively presented in the Modern Times, the Charlie Chaplin film.
\end{note}

\subsubsection{Predictive machines}

\textbf{Numerical data constitute the material of predictive machines}, now considered to replace the judgment
of the magistrate, transmuted into the robo-judge.

\subsection{Quality and Quantity: people and data in the network}

Similary convincing is the claim that ...
\begin{claim}
    Playing with children, caring for the elderly, and many other actions that involve social interaction
    will be inherently better when perfomed by a human being.
\end{claim}

These theses are supported in sharing the analysis that shows that social interaction are fully such when 
they take place between human beings, between people who find each other in the reciprocity that is not
mastered by those who are more powerful in calculating = scheming.

The person does not find himself, does not recognize himself in a robot, who is constitutively devoid
of virtues and vices, does not have serenity, love, generosity, etc., nor does he have anger, hatred, greed, etc. 
He is anaffective, without pathos, without heart, as one might say with Schopenhauer.

As for legal judgment, the use, in the data civilization, of automating tools - algorithms, software, 
\textbf{predictive machines}, etc. - invites the consideration that "robogiudists in principle could ensure
that, for the firs time in history, everyone is truly equal before the law".

The judge, who became a "robojudicator", \unsure{Ha scritto 'robogijudicator.. errore di battitura probabilmente} rooted in the "robolegislator"
, would not render \textbf{judgments corrupted} by means, by knowledge of data, imperfect, biased, but would
pronunce \textbf{verdicts} which have the structure of mathematical certainity.

\begin{example}
    Reference is made to use of equipment such as scanner and magnetic resonance imaging, which should
    be able acquire knowledge of what is the objective-material content of brain operations, so as to make
    false testimony, for example, impossible.
\end{example}

It should be considered that... the \textbf{content} and \textbf{meaning} of what is \textbf{thought} do not
consist solely of the set of elements of brain operations that can be objectified by the instruments employed
by the robographer, since one has in mind that the individual human being, treated by an apparatus deemed suitable
for acquiring knowledge of what he thinks, could decide, by his free will, to think only what has no negative
bearing on the machinic formation of the robographer's judgment.

Such an individual human being could choose to recite a set of thoughts that is presented
as integrally true in its objectification acquired by an artificial intelligence that obtains brain data,
but instead is solely an act.

One has here what happens on a theatrical stage where the actor thinks what he says but immersing 
himself in a character that is not his own person.

Some questions are inevitable : 

\begin{question}
The intelligent machine is capable of making the distinction between that the human being thinks and 
what he says by his actions?
\end{question}

\begin{question}
Is it possible to present such actions to the observation of robojudge ?
\end{question}

\begin{question}
The robojudge can issue a sentence condemning him?
\end{question}

Suc a distinction would be possible if they oculd be objected and separated into 
\begin{enumerate}
    \item The will that chooses to think what he thinks.
    \item The will that decides to 'act' what he thinks. 
\end{enumerate}


This objectification-separation is not practicable, because in every human being his or her
self consists of a simultaneous, interwoven \unsure{interwoven errore di battitura?} multiplicity of
directions of thought, with the consequence that every self does not only think what it thinks,
but in its thinking there are referrals to thoughts in the making, not yet pre-calculable, even 
unconscious, and therefore not objectifiable.

The selection of the purposes of a legal system should currently belong to the figure of the \textbf{robolegislators}, who would
operate according to the proper procedure of algorithms, artifical intelligence, etc... \textit{so as to produce-institute a 
mathematically correct set of norms.}


\textbf{Robojudge} and \textbf{robolegislator} are situated in front of the alternative 'to function or to exist'
they can either operate in the sole itinerary of functional, mathematically correct success, or they can 
engage existence, in the entirety of their inner life, relating to the inner life of other human beings.

Priority is given to operation, having in mind that \textbf{operation} means only everything that can be
reduced by data to information!

\begin{example}
    In human beings, the gaze is not limited to the organ of sight, but constitutes, from its first operation,
    already a search, an interpretation of meaning, unlike the seeing of other living beings or the machinic seeing 
    of viewers... for example, of those instruments employed in the recording of images.
\end{example}

The \textbf{robojudge} can enunciate norms in their presentation according to the patterns of \textbf{computational thinking}... he cannot
enunciate law, cannot therefore grapple with the difference that separates and units \textbf{norms} and texbf{law}; the \textbf{text of laws} is 
oriented toward law. 

No \textbf{robojudge} can accomplish a work that is proper to the art of interpretation! \textbf{The robojudge} cannot \textit{unite} and 
\textit{distinguish} \textbf{norms} and \textbf{law}, the certainty of legality and the anxiety of justice,
never suppressible in the inner life of human beings. 


The figures of the \textbf{robojudge} and the \textbf{robolegislator} are clarified by the analysis of the figure 
of the robophilosopher, giving light to the condition and task of contemporary Philosophy of Law.

The \textbf{robophilosopher} is built and programmed in such ways that should enable him to possess the data
consisting of the knowledge of philosophizing of all times and places. 

On this direction, who better than the robo-philoshoper, "who better than an artificial intelligence,
with the breadth of its knowledge, the rigor of its reasoning and the disinterestedness of its judgment"
could operate in that terrain that has so far been left to the limitedness of human beings.

The robo-philosopher is programmed to acquire and process data referable to questions of 'philosophizing',even those
that belong to Hegel, such as the servant-master dialectic presented in the Phenomenology of Spirit and now central
to the data civilization.

The robo-philosopher is programmed to acquire and process data referable to questions of 'philosophizing', even those 
to the data civilization.


The contemporary situation is precisely qualified by ... 
\begin{itemize}
    \item The emergence of the relations between the lords of the net.
    \item The masters of search engines, platforms, etc.
    \item And the rest of humanity, constituted by those who access the navigation of the \textbf{infosphere}, undergoing the procedures that impose on them the formations of profiles, intended to make them operate as obsequious consumers, effective servants.
\end{itemize}

This new configuration of \textbf{servants} captures the features of automation mechanisation, describing the \textbf{current data civilization}.

The introduction of artificial intelligence, in the sectors of human society, gives a new scope to reflections on the dialect of the 
\textbf{master-slave dialectic}!.

\begin{claim}
    The \textbf{robojudge}, the \textbf{robolegislator} and the \textbf{robo-philosopher}, as machine
    entities and not human beings in the flesh, are not born and do not die, but begin to function and 
    end up functioning.
\end{claim}

\begin{claim}
    The figures of the \textbf{robots} do not dialogue but \textbf{execute} a program built algorithmically 
    to arrive at a \textbf{mathematically correct result} and as such placed outside the circuit of the dialogue.
\end{claim}